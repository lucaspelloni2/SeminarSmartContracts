\chapter{Security Challenges in Smart Constracts}
\markboth{Title of My Seminar Work}{}
\chaptauthors{Lucas Pelloni and Ile Cepilov}

\Kurzfassung{%
This is the abstract.
It fits pretty much on one page and is definitely not longer.}

\newpage

\minitoc %table of contents

\newpage
%http://blockgeeks.com/guides/what-is-blockchain-technology/
%http://www.investopedia.com/terms/b/blockchain.asp
%https://letstalkpayments.com/an-overview-of-blockchain-technology
% PWC: PAPER DA PRENDERE ASSOLUTAMENTE http://www.pwc.ch/en/2017/pdf/pwc_blockchain_opportunity_for_energy_producers_and_consumers_en.pdf


\section{Blockchain Technology}
\subsection{Definition of Blockchain}
A blockchain is like a distributed database which constantly keeps a list of transaction or records, that have ever been executed, called usually blocks. Every record contains a reference to its predecessor and a timestamp. \cite{wikipedia1}
Data in a record cannot be changed due to its design. This makes them tampering-proof and a very good source of trust for the near future. \cite{blockchain3}
The blockchain works and is maintained by the entire community, which verifies all records and acts like a node of a peer-to-peer network, making the need of a thirty part trust organisation useless. \cite{blockchain0}
\subsubsection{Source of Trust}
%http://www.investopedia.com/terms/b/blockchain.asp
Blockchain will become a global decentralised source of trust.
Everything that is centralized makes it easy to attack because it offers a single point of failure. 
(e.g. Firewall of a website). 
Application built with Bloch chain technlogy do not require  users to trust the developers with personal information or funds. 
\subsubsection{How does a Blockchain work?}
           \begin{figure}[ht]
         \begin{center}
         \includegraphics[scale=0.6]{Talk3/blockchain}
         \end{center}
         \caption{This is a pic FROM ILIA}
         \label{label}
       \end{figure}
   
%https://www2.deloitte.com/nl/nl/pages/innovatie/artikelen/blockchain-technology-9-benefits-and-7-challenges.html
\subsection{Benefits of Blockchain Technology}
Every technology that is centralized offers somewhere a point of failure which might be exploited by attackers and nowadays, banks are normally used by most people as a trusted middleman in order to make transaction.
Thanks to its decentralizations, blockchain will avoid the need of a trusted middleman to make transaction by connecting directly buyers and sellers.
%TODO

%http://www.huffingtonpost.com/ameer-rosic-/5-blockchain-applications_b_13279010.html
\subsection{Blockchain applications}
The most famous application of blockchain is in online trades where it gets used for transfering a cryptovalue, bitcoins.
%TODO 


%https://www.shapingtomorrow.com/home/alert/665529-Future-of--Blockchain
%http://usblogs.pwc.com/emerging-technology/the-blockchain-problem-is-a-trust-problem/
\subsection{Future of Blockchain}
%TODO 

%https://www.taylorwessing.com/download/article-how-secure-is-block-chain.html
\subsection{Blockchain security}
%TODO 


\section{Smart Contract: Introduction}
\subsection{Smart Contract: a self-executing contractual agreements}
%definition founded here: https://www.youtube.com/watch?v=FkeLDPZ-v8g&t=134s
A Smart Contract is a computer protocol (or piece of software) that includes, simplifies and verifies the negotiating terms for the execution of a contract. They do not contain a formal contract clause but they usually simulate it.
Smart Contracts are encrypted and implemented as programs running on the nodes of a custom public blockchain network, which is not the same used for Bitcoins transactions. Thanks to their encryption, an external trusted authority is no longer needed and thanks to the redundancy of the blockchain across the network, there will not be any loss of data or integrity in case of a system failure.
If a new Smart Contract has to be added to the blockchain, the miners, which are the actual nodes of the peer to peer network, have first to validate its execution and only once it has been validated by the majority of them, it can be actually appended to the blockchain.
%https://bitsonblocks.net/2016/02/01/a-gentle-introduction-to-smart-contracts/%
\subsubsection{Traditional vs. Smart Contract}

\subsection{Ethereum: a public blockchain-based platform}
Ethereum is a platform for the publication and the execution of Smart Contracts. It is decentralized INSERIRE PAPER and executes the Smart Contracts as programmed from their creator.
It has a own cryptovalue called Ether which is used as payment for the execution of Smart Contracts.
At the 24th of April 2017, the value of the currency compared to Bitcoins corresponded to CONTINUA
Ethereum's contracts are written in Solidity, which is an objected-oriented language programming language, similar to Java, and it is exclusively designed for coding Smart Contracts.
The Smart Contracts are then compiled in the Ethereum virtual machine (EVM bytecode) and deployed finally on the Ethereum blockchain 

\subsection{Smart Contract's properties in Ethereum}
In this section we will have a look at the key components that can be found inside a Smart Contract.



\subsubsection{Solidity: a typed JavaScript-like language}


\section{Smart Contract attacks}

\subsection{DAO: First Attack}
\subsubsection{Attack description}
\subsubsection{Exploited vulnerability}
\subsubsection{Possible solution}

\subsection{DAO: Second Attack}
\subsubsection{Attack description}
\subsubsection{Exploited vulnerability}
\subsubsection{Possible solution}

\subsection{King of The Ether Throne Attack}
\subsubsection{Attack description}
\subsubsection{Exploited vulnerability}
\subsubsection{Possible solution}

\subsection{MultiPlayer Attack}
\subsubsection{Attack description}
\subsubsection{Exploited vulnerability}
\subsubsection{Possible solution}

\subsection{Rubixi Attack}
\subsubsection{Attack description}
\subsubsection{Exploited vulnerability}
\subsubsection{Possible solution}

\subsection{Govern-Mental: First Attack}
Blablabla said asfbzasfubasbusauhfhu \cite{WinNT}.
\subsubsection{Attack description}
\subsubsection{Exploited vulnerability}
\subsubsection{Possible solution}

\subsection{Govern-Mental: Second Attack}
\subsubsection{Attack description}
\subsubsection{Exploited vulnerability}
\subsubsection{Possible solution}

%http://usblogs.pwc.com/emerging-technology/tag/blockchain/
%http://www.coindesk.com/blockchain-smart-contracts-looming-challenges/
\section{Security Challenges in Smart Contract}
\subsection{No guarantee of transaction execution}
\subsection{Slowness of Smart Contract}
\subsection{Code is slow}

\begin{thebibliography}{99}
\bibitem{ethereum1}\emph{Ethereum Homestead Release.} \url{https://www.ethereum.org} (last accessed March 2017)
\bibitem{wikipedia1}\emph{Blockchain} Wikipedia: \url{https://en.wikipedia.org/wiki/Blockchain} (last accessed February 2017)
\bibitem{blockchain1}\emph{Know More About Blockchain.} \url{https://letstalkpayments.com/an-overview-of-blockchain-technology/} (last accessed March 2017)
\bibitem{blockchain2}\emph{Blockchain Use Cases Part II.} \url{https://letstalkpayments.com/blockchain-use-cases-part-ii-non-financial-and-financial-use-cases/} (last accessed March 2017)


\bibitem{paper1}Loi Luu, Duc-Hiep Chu, Hrishi Olickel, Prateek Saxena, Aquinas Hobor: \emph{Making Smart Contracts Smarter.}. October 2016. \url{http://delivery.acm.org/10.1145/2980000/2978309/p254-luu.pdf?ip=195.176.96.218&id=2978309&acc=ACTIVE\%20SERVICE&key=FC66C24E42F07228\%2EA04051DB0C098788\%2E4D4702B0C3E38B35\%2E4D4702B0C3E38B35&CFID=926344688&CFTOKEN=88107312&__acm__=1492712435_d0f0d96ea04f8cf077c9b90253210778} (last accessed February 2016)

\bibitem{paper1}Atzei, Nicola and Bartoletti, Massimo and Cimoli, Tiziana: \emph{A survey of attacks on Ethereum smart contracts.} 2016. Cryptology ePrint Archive: Report 2016/1007, https://eprint. iacr. org/2016/1007 \url{https://eprint.iacr.org/2016/1007.pdf} (last accessed February 2017)

\bibitem{blockchain3}Investopedia: \emph{What is a Blockchain.} \url{http://www.investopedia.com/terms/b/blockchain.asp} (last accessed March 2017)

\bibitem{blockchain4}\emph{Blockchain technology: 9 benefits and 7 challenges.} \url{https://www2.deloitte.com/nl/nl/pages/innovatie/artikelen/blockchain-technology-9-benefits-and-7-challenges.html} (last accessed March 2017)

\bibitem{blockchain5}Hasse, von Perfall, Hillebrand, Smole, Lay, Charlet: \emph{Blockchain - an opportunity for energy producers and consumers?.}\url{http://www.pwc.ch/en/2017/pdf/pwc_blockchain_opportunity_for_energy_producers_and_consumers_en.pdf} (last accessed March 2017)

\bibitem{blockchain6}\emph{What is Blockchain Technology?.} \url{https://blockgeeks.com/guides/what-is-blockchain-technology/} (last accessed March 2017)
\bibitem{blockchain7}\emph{Know More About Blockchain.} \url{https://letstalkpayments.com/an-overview-of-blockchain-technology/} (last accessed April 2017)
\bibitem{blockchain8}\emph{Five Blockchain Applications That Are Shaping Your Future.}\url{http://www.huffingtonpost.com/ameer-rosic-/5-blockchain-applications_b_13279010.html} (last accessed March 2017)
\bibitem{blockchain9}\emph{Future of Blockchain.} \url{https://www.shapingtomorrow.com/home/alert/665529-Future-of--Blockchain} (last accessed February 2017)
\bibitem{blockchain10}\emph{The blockchain problem is a trust problem.}\url{http://usblogs.pwc.com/emerging-technology/the-blockchain-problem-is-a-trust-problem/} (last accessed February 2017)
\bibitem{blockchain11}\emph{How secure is blockchain?.} \url{https://www.taylorwessing.com/download/article-how-secure-is-block-chain.html} (last accessed March 2017)
\bibitem{SC1}\emph{Simple introduction to smart contracts on a blockchain.} \url{https://www.youtube.com/watch?v=FkeLDPZ-v8g&t=134s} (last accessed March 2017)
\bibitem{SC2}\emph{A gentle introduction to smart contracts.} \url{https://bitsonblocks.net/2016/02/01/a-gentle-introduction-to-smart-contracts/} (last accessed February 2017)
\bibitem{SC3}\emph{Blockchain Pros Debate Looming Challenges for Smart Contracts.} \url{http://www.coindesk.com/blockchain-smart-contracts-looming-challenges/} (last accessed April 2017)



\bibitem{blockchain0}Morabito: \emph{Blockchain Value System.} 2017. \url{http://www.springer.com/cda/content/document/cda_downloaddocument/9783319484778-c2.pdf?SGWID=0-0-45-1599947-p180347565} (last accessed March 2017)
\bibitem{SC5}\emph{title.}\url{url} (last accessed April 2017)
\bibitem{SC6}\emph{title.}\url{url} (last accessed April 2017)
\bibitem{SC7}\emph{title.}\url{url} (last accessed April 2017)
\bibitem{SC8}\emph{title.}\url{url} (last accessed April 2017)
\bibitem{SC9}\emph{title.}\url{url} (last accessed April 2017)
\bibitem{SC10}\emph{title.}\url{url} (last accessed April 2017)
\bibitem{SC11}\emph{title.}\url{url} (last accessed April 2017)
\bibitem{SC12}\emph{title.}\url{url} (last accessed April 2017)
\bibitem{SC13}\emph{title.}\url{url} (last accessed April 2017)









\end{thebibliography}
